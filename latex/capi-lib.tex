\documentclass[landscape,twocolumn,10pt,a4paper]{article}
\usepackage{graphicx} 
\usepackage{subfiles} 
\usepackage{titlesec}
\usepackage{minted}

\usepackage{geometry}
\geometry{%
  top=1.5cm, 
  bottom=1.5cm, 
  inner=1.5cm, 
  outer=1.5cm, 
  heightrounded, 
  bindingoffset=0.5cm % Espaço adicional para a encadernação, se necessário
}

\setlength{\columnseprule}{0.4pt}
\setminted{
    breaklines=true,      % Quebra de linha automática
    breakanywhere=true,
    breaksymbolleft={},   % Sem símbolo de quebra de linha
    breakindent=0pt,      % Sem indentação nas linhas quebradas
    breaksymbolsepleft=0pt, % Sem separação para o símbolo de quebra
}


\title{Capi-Library
}
\author{Bruno Rezendo, Dener Ribeiro, Samuel Raimundo}
% \date{May 2024}

\begin{document}

\maketitle

\tableofcontents\section{Estruturas}
\subsection{waveletTree}
\begin{minted}{cpp}

#include "bits/stdc++.h"

using namespace std;

struct waveletTree {
    int lo, hi; // Range of elements
    waveletTree *l, *r; // Left and Right child
    vector<int> b; // Freq array

    // Array is in range [x, y]
    // Indices are in range [from, to]
    waveletTree(vector<int>::iterator from, vector<int>::iterator to, int x, int y) {
        lo = x, hi = y;

        // Array is of 0 length
        // Or a is homogeneous, like {1,1,1,1}
        if(from == to or hi == lo) return;

        b.reserve(to - from + 1);
        b.push_back(0);

        int mid = (lo+hi)/2;
        // Lambda function to check if a number
        // is less than or equal to mid
        auto f = [mid](int x){
			return x <= mid;
		};

        for(auto it = from; it != to; it++)
			b.push_back(b.back() + f(*it));

        auto pivot = stable_partition(from, to, f);
		l = new waveletTree(from, pivot, lo, mid);
		r = new waveletTree(pivot, to, mid+1, hi);
    }

    // count of elements in [l, r] equal to k
    // the interval is 1-indexed
	int count(int l, int r, int k) {
		if(l > r or k < lo or k > hi) return 0;
		if(lo == hi) return r - l + 1;
		int lb = b[l-1], rb = b[r], mid = (lo+hi)/2;
		if(k <= mid) return this->l->count(lb+1, rb, k);
		return this->r->count(l-lb, r-rb, k);
	}

    // kth smallest element in [l, r]
    // the interval is 1-indexed
	int kth(int l, int r, int k){
		if(l > r) return 0;
		if(lo == hi) return lo;
		int lb = b[l-1]; //amt of nos in first (l-1) nos that go in left 
		int rb = b[r]; //amt of nos in first (r) nos that go in left
		int inLeft = rb - lb;
		if(k <= inLeft) return this->l->kth(lb+1, rb , k); // kth is in the left child
		return this->r->kth(l-lb, r-rb, k-inLeft); // kth is the in right child
	}

    // count of nos in [l, r] that are <= k
    // the interval is 1-indexed
	int lte(int l, int r, int k) {
		if(l > r or k < lo) return 0;
		if(hi <= k) return r - l + 1;
		int lb = b[l-1], rb = b[r];
		return this->l->lte(lb+1, rb, k) + this->r->lte(l-lb, r-rb, k);
	}
    
    // swap element with index (i) with element with index (i+1)
    // the index is 1-indexed
    void swap(int i) {
        if (lo == hi or i >= b.size() or i <= 0) return;
        bool iLeft = b[i] > b[i-1];
        bool i1Left = b[i+1] > b[i];
        if (iLeft && i1Left) this->l->swap(b[i]);
        if (!iLeft && !i1Left) this->r->swap(i-b[i]);
        if (iLeft && !i1Left) b[i]--;
        if (!iLeft && i1Left) b[i]++;
    }
};

int main() {
    int lo = INT_MAX, hi = INT_MIN;
    vector<int> v = {3,1,7,5,2,6,4,8,1,2};
    for (auto el : v) {
        lo = min(lo, el);
        hi = max(hi, el);
    }

    waveletTree wt(v.begin(), v.end(), lo, hi);
    cout << wt.count(1, 10, 1) << endl;
    cout << wt.count(7, 9, 1) << endl;
    cout << wt.kth(1, 10, 3) << endl;
    cout << wt.kth(7, 10, 3) << endl;
    cout << wt.lte(1, 10, 2) << endl;
    cout << wt.lte(7, 10, 2) << endl;

    return 0;
}\end{minted}

\section{Grafos}
\subsection{articulationPoints}
\begin{minted}{cpp}

#include "bits/stdc++.h"

#define ll long long

using namespace std;

const ll MAXN = 200010;
vector<vector<int>> graph(MAXN);
vector<bool> visited(MAXN);
vector<int> articulationPoints;
vector<int> tin(MAXN);
vector<int> lowlink(MAXN);
int entryTime;

void init(int numV, int numE) {
    entryTime = 0;
    articulationPoints.clear();
    for (int i = 0; i < numV; i++) {
        graph[i].clear();
        visited[i] = false;
    }
}

void dfs(int v, int parent) {
    visited[v] = true;
    lowlink[v] = tin[v] = entryTime++;

    bool isArticulationPoint = false;
    int numChildren = 0;

    for (int u : graph[v]) {
        if (!visited[u]) {
            numChildren++;
            
            dfs(u, v);
            lowlink[v] = min(lowlink[v], lowlink[u]);

            if (lowlink[u] >= tin[v]) isArticulationPoint = true;
        } else {
            if (u == parent) continue;

            lowlink[v] = min(lowlink[v], tin[u]);
        }
    }

    if (v == parent && numChildren >= 2) articulationPoints.push_back(v);
    if (v != parent && isArticulationPoint) articulationPoints.push_back(v);
}

void solve() {
    ll n, m, a, b;
    cin >> n >> m;
    init(n, m);

    for (ll i = 0; i < m; i++) {
        cin >> a >> b; a--; b--;
        graph[a].push_back(b);
        graph[b].push_back(a);
    }

    for (int i = 0; i < n; i++) if(!visited[i]) dfs(i,i);

    cout << "Number of articulation points: " << articulationPoints.size() << endl;
    for (int point : articulationPoints) cout << point+1 << " ";
    cout << endl;
}

int main() {
    ll t;
    cin >> t;

    while (t--) solve();

    return 0;
}\end{minted}

\subsection{bridgeTreeEbcc}
\begin{minted}{cpp}

#include "bits/stdc++.h"

#define ll long long

using namespace std;

const ll MAXN = 200010;
vector<vector<int>> graph(MAXN);
vector<bool> visited(MAXN);
vector<int> tin(MAXN);
vector<int> lowlink(MAXN);
stack<int> vertices;
vector<vector<int>> ebcc; // edge-biconnected components
int entryTime;

void init(int numV, int numE) {
    entryTime = 0;
    ebcc.clear();
    for (int i = 0; i < numV; i++) {
        graph[i].clear();
        visited[i] = false;
    }
}

void dfs(int v, int parent) {
    visited[v] = true;
    lowlink[v] = tin[v] = entryTime++;
    vertices.push(v); 

    for (int u : graph[v]) {
        if (!visited[u]) {
            dfs(u, v);
            lowlink[v] = min(lowlink[v], lowlink[u]);
        } else {
            if (u == parent) continue;

            lowlink[v] = min(lowlink[v], tin[u]);
        }
    }

    // we need to execute this for the root, so there's no need to check (v != parent)
    if (lowlink[v] == tin[v]) {
        vector<int> newComponent;
        do {
            newComponent.push_back(vertices.top());
            vertices.pop();
        } while (newComponent.back() != v);

        ebcc.push_back(newComponent);
    }
}

void solve() {
    ll n, m, a, b;
    cin >> n >> m;
    init(n, m);

    for (ll i = 0; i < m; i++) {
        cin >> a >> b; a--; b--;
        graph[a].push_back(b);
        graph[b].push_back(a);
    }

    // DOES NOT WORK WITH MULTIPLE EDGES
    for (int i = 0; i < n; i++) if(!visited[i]) dfs(i,i);

    cout << "Num components: " << ebcc.size() << endl; 
    for (auto component : ebcc) {
        cout << "=> ";
        for (auto el : component) cout << el+1 << " ";
        cout << endl;
    }       
}

int main() {
    ll t;
    cin >> t;

    while (t--) solve();

    return 0;
}\end{minted}

\subsection{biconnectedComponents}
\begin{minted}{cpp}

// Biconnected Components = 2-Vertex Connected Components
#include "bits/stdc++.h"

#define ll long long

using namespace std;

const ll MAXN = 200010;
vector<vector<int>> graph(MAXN);
vector<vector<int>> edgeIds(MAXN);
vector<pair<int,int>> edges(MAXN);
vector<bool> visited(MAXN);
vector<bool> edgeVisited(MAXN);
vector<int> tin(MAXN);
vector<int> lowlink(MAXN);
vector<int> isArticulationPoint(MAXN);
stack<int> edgesStack;
vector<vector<pair<int,int>>> bcc; // biconnected components
int entryTime;

void init(int numV, int numE) {
    entryTime = 0;
    bcc.clear();
    for (int i = 0; i < numV; i++) {
        graph[i].clear();
        edgeIds[i].clear();
        visited[i] = false;
        isArticulationPoint[i] = false;
    }
    for (int i = 0; i < 2*numE; i++) {
        edgeVisited[i] = false;
    }
}

void newBiconnectedComponent (int edgeId) {
    if(edgesStack.empty()) return; // dealing with connected component of one vertex
    
    vector<pair<int,int>> newComponent;
    int currId;
    do {
        currId = edgesStack.top();
        newComponent.push_back(edges[currId]);
        edgesStack.pop();
        if(edgesStack.empty()) break;
    } while(currId != edgeId);
    bcc.push_back(newComponent);
}

void dfs(int v, int parent) {
    visited[v] = true;
    lowlink[v] = tin[v] = entryTime++;

    int numChildren = 0;
    for (int i = 0; i < graph[v].size(); i++) {
        int u = graph[v][i];
        int edgeId = edgeIds[v][i];

        if (edgeVisited[edgeId]) continue;
        edgeVisited[edgeId] = true;
        edgesStack.push(edgeId);

        if (!visited[u]) {
            numChildren++;

            dfs(u, v);
            lowlink[v] = min(lowlink[v], lowlink[u]);

            bool foundNewComponent = false;
            if (v == parent && numChildren >= 2) foundNewComponent = true;
            if (v != parent && lowlink[u] >= tin[v]) foundNewComponent = true;

            if (foundNewComponent) {
                isArticulationPoint[v] = true;
                newBiconnectedComponent(edgeId);
            }
        } else {
            lowlink[v] = min(lowlink[v], tin[u]);
        }
    }
}

void findBiconnectedComponents(int n) {
    for (int i = 0; i < n; i++) {
        if(!visited[i]) {
            dfs(i,i);
            newBiconnectedComponent(-1);
        }
    }
}

void solve() {
    ll n, m, a, b;
    cin >> n >> m;
    init(n, m);

    for (ll i = 0; i < m; i++) {
        cin >> a >> b; a--; b--;
        graph[a].push_back(b);
        graph[b].push_back(a);
        edgeIds[a].push_back(i);
        edgeIds[b].push_back(i);
        edges[i] = {a, b};
    }

    findBiconnectedComponents(n);

    cout << "Num components: " << bcc.size() << endl; 
    for (auto component : bcc) {
        cout << "=> ";
        for (auto el : component) cout << "(" << el.first+1 << ", " << el.second+1 << ") ";
        cout << endl;
    }       
}

int main() {
    ll t;
    cin >> t;

    while (t--) solve();

    return 0;
}\end{minted}

\subsection{smallToLarge}
\begin{minted}{cpp}

// Small To Large
// 1. Faz queries sem update na subárvore
// 2. Resolve os filhos leves e depois os pesados. 
// 3. A resposta para os filhos leves é sempre adicionada e apagada. 
// 4. A resposta dos filhos pesados é mantida.
//
// O(N * log(N) * C(u))
// 
// C(u) é acomplexidade do update da
// informação de um vértice na resposta

void add(int u) {/* aqui adiciona a subarvore de u na resposta */}

void rem(int u) {/* aqui remove a subarvore de u da resposta */}
 
void solve(int u, int keep=1) {
    int mx = -1;
    for(int v : adj[u]) if(v != par[u]) {
        if(mx == -1 or subt[v] >= subt[mx]) {
            mx = v;
        }
    }
    for(int v : adj[u]) if(v != par[u] and v != mx) solve(v, 0);
    if(mx != -1) solve(mx, 1);
    /* aqui adiciona a informação de u na resposta */
    for(int v : adj[u]) if(v != par[u] and v != mx) add(v);
    /* aqui responde a query de u. ans[u] = ... */
    if(!keep) rem(u);
}\end{minted}

\subsection{2sat}
\begin{minted}{cpp}

#include "bits/stdc++.h"

#define ll long long

using namespace std;

const ll MAXN = 200010;
vector<vector<int>> graph(MAXN);
vector<vector<int>> reverseGraph(MAXN);
vector<int> visited(MAXN);
vector<int> exitOrder;
vector<int> componentOfV(MAXN);
vector<bool> res(MAXN);
vector<vector<int>> components;

void init(int numV, int numE) {
    exitOrder.clear();
    components.clear();
    for (int i = 0; i < numV; i++) {
        graph[i].clear();
        reverseGraph[i].clear();
        visited[i] = 0;
    }
}

void dfs(int v) {
    visited[v] = 1;

    for (int u : graph[v]) if (visited[u] == 0) dfs(u);

    exitOrder.push_back(v);
}

void dfs2(int v, int c) {
    visited[v] = 2;
    componentOfV[v] = c;
    components.back().push_back(v);

    for (int u : reverseGraph[v]) if (visited[u] == 1) dfs2(u, c);
}

void addDisjunction(int a, bool na, int b, bool nb) {
    // na and nb signify whether a and b are to be negated 
    a = 2*a ^ na;
    b = 2*b ^ nb;
    int negA = a ^ 1;
    int negB = b ^ 1;

    graph[negA].push_back(b);
    graph[negB].push_back(a);
    reverseGraph[b].push_back(negA);
    reverseGraph[a].push_back(negB);
}

bool solve2Sat(int n) {
    // first dfs of kosaraju
    for (int i = 0; i < 2*n; i++) if(visited[i] == 0) dfs(i);

    int componentId = 0;
    for (int i = exitOrder.size()-1; i >= 0; i--) if(visited[exitOrder[i]] == 1) {
        components.push_back(vector<int>());
        dfs2(exitOrder[i], componentId++);
    }

    for (int i = 0; i < n; i++) {
        if (componentOfV[2*i] == componentOfV[2*i+1]) return false;

        res[i] = (componentOfV[2*i] > componentOfV[2*i+1]);
    }
    return true;
}

void solve() {
    ll n, m, a, b;
    char c1, c2;

    cin >> n >> m;

    init(2*n, 2*m);

    for (ll i = 0; i < m; i++) {
        cin >> c1 >> a >> c2 >> b; a--; b--;
        
        addDisjunction(a, c1 == '-', b, c2 == '-');
    }

    if (solve2Sat(n)) {
        for (int i = 0; i < n; i++) {
            cout << (res[i] ? "+" : "-") << " \n"[i == n-1];
        }
    } else {
        cout << "IMPOSSIBLE" << endl;
    }
}

int main() {
    ll t;
    cin >> t;

    while (t--) solve();

    return 0;
}\end{minted}

\subsection{condensationGraph}
\begin{minted}{cpp}

#include "bits/stdc++.h"

#define ll long long

using namespace std;

const ll MAXN = 200010;
vector<vector<int>> graph(MAXN);
vector<vector<int>> reverseGraph(MAXN);
vector<int> visited(MAXN);
vector<int> exitOrder;
vector<int> componentOfV(MAXN);
vector<unordered_set<int>> condensationGraphEdges(MAXN);
vector<vector<int>> condensationGraph(MAXN);
vector<vector<int>> components;

void init(int numV, int numE) {
    exitOrder.clear();
    components.clear();
    for (int i = 0; i < numV; i++) {
        graph[i].clear();
        reverseGraph[i].clear();
        condensationGraph[i].clear();
        condensationGraphEdges[i].clear();
        visited[i] = 0;
    }
}

void dfs(int v) {
    visited[v] = 1;

    for (int u : graph[v]) if (visited[u] == 0) dfs(u);

    exitOrder.push_back(v);
}

void dfs2(int v, int c) {
    visited[v] = 2;
    componentOfV[v] = c;
    components.back().push_back(v);

    for (int u : reverseGraph[v]) if (visited[u] == 1) dfs2(u, c);
}

void dfs3(int v, int parent) {
    if (componentOfV[v] != componentOfV[parent]) {
        condensationGraphEdges[componentOfV[parent]].insert(componentOfV[v]);
    }
    visited[v] = 3;

    for (int u : graph[v]) if (visited[u] == 2) dfs3(u, v);
}

void createCondensationGraph(int n) {
    for (int i = 0; i < n; i++) if(visited[i] == 2) dfs3(i,i);

    for (int i = 0; i < components.size(); i++) {
        condensationGraph[i].insert(condensationGraph[i].end(), condensationGraphEdges[i].begin(), condensationGraphEdges[i].end());
    }
}

void solve() {
    ll n, m, a, b;
    cin >> n >> m;
    init(n, m);

    for (ll i = 0; i < m; i++) {
        cin >> a >> b; a--; b--;
        graph[a].push_back(b);
        reverseGraph[b].push_back(a);
    }

    for (int i = 0; i < n; i++) if(visited[i] == 0) dfs(i);

    int componentId = 0;
    for (int i = exitOrder.size()-1; i >= 0; i--) if(visited[exitOrder[i]] == 1) {
        components.push_back(vector<int>());
        dfs2(exitOrder[i], componentId++);
    }

    createCondensationGraph(n);

    cout << "Condensation graph num of vertices: " << components.size() << endl;
    for (int i = 0; i < components.size(); i++) {
        for (int j = 0; j < condensationGraph[i].size(); j++) {
            cout << "(" << i+1 << ", " << condensationGraph[i][j]+1 << ") ";
        }
    }
    cout << endl << endl;
}

int main() {
    ll t;
    cin >> t;

    while (t--) solve();

    return 0;
}\end{minted}

\subsection{bridges}
\begin{minted}{cpp}

#include "bits/stdc++.h"

#define ll long long

using namespace std;

const ll MAXN = 200010;
vector<vector<int>> graph(MAXN);
vector<bool> visited(MAXN);
vector<int> tin(MAXN);
vector<int> lowlink(MAXN);
vector<pair<int, int>> bridges;
int entryTime;

void init(int numV, int numE) {
    entryTime = 0;
    bridges.clear();
    for (int i = 0; i < numV; i++) {
        graph[i].clear();
        visited[i] = false;
    }
}

void dfs(int v, int parent) {
    visited[v] = true;
    lowlink[v] = tin[v] = entryTime++;

    for (int u : graph[v]) {
        if (!visited[u]) {
            dfs(u, v);
            lowlink[v] = min(lowlink[v], lowlink[u]);
        } else {
            if (u == parent) continue;

            lowlink[v] = min(lowlink[v], tin[u]);
        }
    }

    if (v != parent && lowlink[v] == tin[v]) {
        bridges.push_back({v, parent});
    }
}

void solve() {
    ll n, m, a, b;
    cin >> n >> m;
    init(n, m);

    for (ll i = 0; i < m; i++) {
        cin >> a >> b; a--; b--;
        graph[a].push_back(b);
        graph[b].push_back(a);
    }

    for (int i = 0; i < n; i++) if(!visited[i]) dfs(i,i);

    cout << bridges.size() << endl;        
}

int main() {
    ll t;
    cin >> t;

    while (t--) solve();

    return 0;
}\end{minted}

\subsection{bridgeTreeEbccMultipleEdges}
\begin{minted}{cpp}

#include "bits/stdc++.h"

#define ll long long

using namespace std;

const ll MAXN = 200010;
vector<vector<int>> graph(MAXN);
vector<vector<int>> edgeIds(MAXN);
vector<bool> visited(MAXN);
vector<bool> edgeVisited(MAXN);
vector<int> tin(MAXN);
vector<int> lowlink(MAXN);
stack<int> vertices;
vector<vector<int>> ebcc; // edge-biconnected components
int entryTime;

void init(int numV, int numE) {
    entryTime = 0;
    ebcc.clear();
    for (int i = 0; i < numV; i++) {
        graph[i].clear();
        edgeIds[i].clear();
        visited[i] = false;
    }
    for (int i = 0; i < 2*numE; i++) {
        edgeVisited[i] = false;
    }
}

void dfs(int v, int parent) {
    visited[v] = true;
    lowlink[v] = tin[v] = entryTime++;
    vertices.push(v); 

    for (int i = 0; i < graph[v].size(); i++) {
        int u = graph[v][i];
        int edgeId = edgeIds[v][i];

        if (edgeVisited[edgeId]) continue;
        edgeVisited[edgeId] = true;

        if (!visited[u]) {
            dfs(u, v);
            lowlink[v] = min(lowlink[v], lowlink[u]);
        } else {
            // with multiple edges, we can go back to the parent via an unvisited edge
            // if (u == parent) continue;

            lowlink[v] = min(lowlink[v], tin[u]);
        }
    }

    // we need to execute this for the root, so there's no need to check (v != parent)
    if (lowlink[v] == tin[v]) {
        vector<int> newComponent;
        do {
            newComponent.push_back(vertices.top());
            vertices.pop();
        } while (newComponent.back() != v);

        ebcc.push_back(newComponent);
    }
}

void solve() {
    ll n, m, a, b;
    cin >> n >> m;
    init(n, m);

    for (ll i = 0; i < m; i++) {
        cin >> a >> b; a--; b--;
        graph[a].push_back(b);
        graph[b].push_back(a);
        edgeIds[a].push_back(i);
        edgeIds[b].push_back(i);
    }

    for (int i = 0; i < n; i++) if(!visited[i]) dfs(i,i);

    cout << "Num components: " << ebcc.size() << endl; 
    for (auto component : ebcc) {
        cout << "=> ";
        for (auto el : component) cout << el+1 << " ";
        cout << endl;
    }       
}

int main() {
    ll t;
    cin >> t;

    while (t--) solve();

    return 0;
}\end{minted}

\subsection{tarjan}
\begin{minted}{cpp}

// Algoritmo de Tarjan para pontes
// O(n+m)

const int MAXN = 200010;

vector<bool> marc(MAXN,false);
vector<int> pre(MAXN),low(MAXN),comp(MAXN);
stack<int> pilha;

vector<int> grafo[MAXN];
vector<int> vertices_da_componente[MAXN];
vector<pair<int,int> > bridges;

vector<int> componente;

int t=0,c=0,p=0;

void dfs(int v,int pai){
    t++;
    pre[v] = t;
    low[v] = t;
    marc[v] = true;
    pilha.push(v);

    for(int viz:grafo[v]){
        if(!marc[viz]){
            dfs(viz,v);
            low[v] = min(low[v],low[viz]);
        }else{
            if(viz==pai) continue;
            low[v] = min(low[v],pre[viz]);
        }
    }

    if(low[v]==pre[v]){
        c++;
        int vertice;

        if(v!=0){
            bridges.push_back({v,pai});
        }

        do{
            vertice = pilha.top();
            pilha.pop();
            comp[vertice] = c;
            vertices_da_componente[c].push_back(vertice);
            componente[vertice] = c;
        }while(vertice != v);
    }
}


vector<vector<int> > bridgesTree(){
    vector<vector<int> > tree(c+1);
    for(int i=0;i<bridges.size();i++){
        auto [a,b] = bridges[i];
        tree[componente[a]].push_back(componente[b]);
        tree[componente[b]].push_back(componente[a]);
    }
    return tree;
}
\end{minted}

\subsection{stronglyConnectedComponents}
\begin{minted}{cpp}

#include "bits/stdc++.h"

#define ll long long

using namespace std;

const ll MAXN = 200010;
vector<vector<int>> graph(MAXN);
vector<vector<int>> reverseGraph(MAXN);
vector<int> visited(MAXN);
vector<int> exitOrder;
vector<int> componentOfV(MAXN);
vector<vector<int>> components;

void init(int numV, int numE) {
    exitOrder.clear();
    components.clear();
    for (int i = 0; i < numV; i++) {
        graph[i].clear();
        reverseGraph[i].clear();
        visited[i] = 0;
    }
}

void dfs(int v) {
    visited[v] = 1;

    for (int u : graph[v]) if (visited[u] == 0) dfs(u);

    exitOrder.push_back(v);
}

void dfs2(int v, int c) {
    visited[v] = 2;
    componentOfV[v] = c;
    components.back().push_back(v);

    for (int u : reverseGraph[v]) if (visited[u] == 1) dfs2(u, c);
}

void solve() {
    ll n, m, a, b;
    cin >> n >> m;
    init(n, m);

    for (ll i = 0; i < m; i++) {
        cin >> a >> b; a--; b--;
        graph[a].push_back(b);
        reverseGraph[b].push_back(a);
    }

    for (int i = 0; i < n; i++) if(visited[i] == 0) dfs(i);

    int componentId = 0;
    for (int i = exitOrder.size()-1; i >= 0; i--) if(visited[exitOrder[i]] == 1) {
        components.push_back(vector<int>());
        dfs2(exitOrder[i], componentId++);
    }

    cout << "Num of components: " << components.size() << endl;
    for (auto component : components) {
        cout << "=> ";
        for (auto el : component) cout << el+1 << " ";
        cout << endl;
    }
}

int main() {
    ll t;
    cin >> t;

    while (t--) solve();

    return 0;
}\end{minted}

\subsection{blockCutTree}
\begin{minted}{cpp}

// Biconnected Components = 2-Vertex Connected Components
#include "bits/stdc++.h"

#define ll long long

using namespace std;

const ll MAXN = 200010;
vector<vector<int>> graph(MAXN);
vector<vector<int>> blockCutTree(MAXN);
vector<vector<int>> edgeIds(MAXN);
vector<pair<int,int>> edges(MAXN);
vector<bool> visited(MAXN);
vector<bool> compVisited(MAXN);
vector<bool> edgeVisited(MAXN);
vector<int> tin(MAXN);
vector<int> lowlink(MAXN);
vector<int> isArticulationPoint(MAXN);
stack<int> edgesStack;
vector<int> componentOfEdge(MAXN);
int componentId;
vector<vector<pair<int,int>>> bcc; // biconnected components
int entryTime;

void init(int numV, int numE) {
    entryTime = 0;
    componentId = 0;
    bcc.clear();
    for (int i = 0; i < numV; i++) {
        graph[i].clear();
        blockCutTree[i].clear();
        edgeIds[i].clear();
        visited[i] = false;
        isArticulationPoint[i] = false;
    }
    for (int i = 0; i < 2*numE; i++) {
        edgeVisited[i] = false;
    }
}

void newBiconnectedComponent (int edgeId) {
    if(edgesStack.empty()) return; // dealing with connected component of one vertex
    
    vector<pair<int,int>> newComponent;
    int currId;
    do {
        currId = edgesStack.top();
        newComponent.push_back(edges[currId]);
        componentOfEdge[currId] = componentId;
        edgesStack.pop();
        if(edgesStack.empty()) break;
    } while(currId != edgeId);
    bcc.push_back(newComponent);
    componentId++;
}

void dfs(int v, int parent) {
    visited[v] = true;
    lowlink[v] = tin[v] = entryTime++;

    int numChildren = 0;
    for (int i = 0; i < graph[v].size(); i++) {
        int u = graph[v][i];
        int edgeId = edgeIds[v][i];

        if (edgeVisited[edgeId]) continue;
        edgeVisited[edgeId] = true;
        edgesStack.push(edgeId);

        if (!visited[u]) {
            numChildren++;

            dfs(u, v);
            lowlink[v] = min(lowlink[v], lowlink[u]);

            bool foundNewComponent = false;
            if (v == parent && numChildren >= 2) foundNewComponent = true;
            if (v != parent && lowlink[u] >= tin[v]) foundNewComponent = true;

            if (foundNewComponent) {
                isArticulationPoint[v] = true;
                newBiconnectedComponent(edgeId);
            }
        } else {
            lowlink[v] = min(lowlink[v], tin[u]);
        }
    }
}

void findBiconnectedComponents(int n) {
    for (int i = 0; i < n; i++) {
        if(!visited[i]) {
            dfs(i,i);
            newBiconnectedComponent(-1);
        }
    }
}

void generateBlockCutTree(int n) {
    for (int i = 0; i < n; i++) {
        if (!isArticulationPoint[i]) continue;
        for (int edgeId : edgeIds[i]) {
            int comp = componentOfEdge[edgeId];
            if (!compVisited[comp]) {
                compVisited[comp] = true;
                blockCutTree[componentId].push_back(comp);
                blockCutTree[comp].push_back(componentId);
            }
        }
        for (int edgeId : edgeIds[i]) {
            int comp = componentOfEdge[edgeId];
            compVisited[comp] = false;
        }
        componentId++;
    }
}

void solve() {
    ll n, m, a, b;
    cin >> n >> m;
    init(n, m);

    for (ll i = 0; i < m; i++) {
        cin >> a >> b; a--; b--;
        graph[a].push_back(b);
        graph[b].push_back(a);
        edgeIds[a].push_back(i);
        edgeIds[b].push_back(i);
        edges[i] = {a, b};
    }

    findBiconnectedComponents(n);
    generateBlockCutTree(n);

    cout << "Num components: " << bcc.size() << endl; 
    for (auto component : bcc) {
        cout << "=> ";
        for (auto el : component) cout << "(" << el.first+1 << ", " << el.second+1 << ") ";
        cout << endl;
    }

    cout << "Block-cut tree num of vertices: " << componentId << endl;
    for (int i = 0; i < componentId; i++) {
        for (int j = 0; j < blockCutTree[i].size(); j++) {
            cout << "(" << i+1 << ", " << blockCutTree[i][j]+1 << ") ";
        }
    }
    cout << endl << endl;
}

int main() {
    ll t;
    cin >> t;

    while (t--) solve();

    return 0;
}\end{minted}

\section{D&C}
\subsection{ternarySearch}
\begin{minted}{cpp}

#include "bits/stdc++.h"
using namespace std;

double fMin(double x){
    return abs(500.0-x);
}

double ternarySearchGetMin(double l, double r) {
    // double eps = 1e-9;              //set the error limit here
    // while (r - l > eps) {
    for (int i = 0; i < 100; ++i) {
        double delta = (r - l) / 3;
        double m1 = l + delta;
        double m2 = r - delta;
        (fMin(m1) > fMin(m2)) ? l = m1 : r = m2;
    }
    return l;
}

double fMax(double x){
    return 1000.0 - abs(500.0-x);
}

double ternarySearchGetMax(double l, double r) {
    // double eps = 1e-9;              //set the error limit here
    // while (r - l > eps) {
    for (int i = 0; i < 100; ++i) {
        double delta = (r - l) / 3;
        double m1 = l + delta;
        double m2 = r - delta;
        (fMax(m1) > fMax(m2)) ? r = m2 : l = m1;
    }
    return l;
}



int main() {
    cout << "Minimum Point: " << ternarySearchGetMin(0.0,1000.0) << endl;
    cout << "Maximum Point: " << ternarySearchGetMax(0.0,1000.0) << endl;
    return 0;
}\end{minted}

\subsection{binarySearch}
\begin{minted}{cpp}

#include "bits/stdc++.h"
using namespace std;

bool p(int x) {
    return x >= 4; // example [false, false, false, false, true, true]
}

int binarySearchFirstTrue(int l, int r) {
    while(l<r) {
        int m = l + (r-l)/2;

        p(m) ? r = m : l = m+1;
    }
    return p(l) ? l : -1;
}

int binarySearchLastFalse(int l, int r) {
    while(l<r) {
        int m = l + (r-l+1)/2;
        p(m) ? r = m-1 : l = m;
    }
    return p(l) ? -1 : l;
}

double binarySearchContinuous(double l, double r) {
    for(int i=0;i<100;i++) {
        double m = l + (r-l)/2;
        p(m) ? r = m : l = m;
    }
    return l; 
}

int main() {
    cout << "First true: " << binarySearchFirstTrue(0,1000) << endl;
    cout << "Last false: " << binarySearchLastFalse(0,1000) << endl;
    return 0;
}\end{minted}

\section{Extra}
\subsection{run}
\begin{minted}{sh}

#!/bin/bash
clear
g++ -std=c++17 -O2 -Wall $2 $1/$1.cpp
for file in $1/in*
do
    echo $file
    ./a.out <$file
    echo "-----------------------"
    echo
done\end{minted}

\subsection{template}
\begin{minted}{cpp}

#include <bits/stdc++.h>

using namespace std;

typedef long long ll;

#define optimize ios_base::sync_with_stdio(false); cin.tie(NULL); cout.tie(NULL);
#define endl "\n"

#define all(x) (x).begin(),(x).end()

const int INF = 0x3f3f3f3f;
const ll LINF = 0x3f3f3f3f3f3f3f3fll;


int main(){
    optimize;


    return 0;
}\end{minted}

\subsection{debug}
\begin{minted}{cpp}

// Debug
// 1. Precisa compilar com -DDEBUG

// para pair
template<typename A, typename B> ostream& operator<<(ostream &os, const pair<A, B> &p) { return os << '(' << p.first << "," << p.second << ')'; }

// para vector,set,...
template<typename T_container, typename T = typename enable_if<!is_same<T_container, string>::value, typename T_container::value_type>::type> ostream& operator<<(ostream &os, const T_container &v) { os << '{'; string sep; for (const T &x : v) os << sep << x, sep = ","; return os << '}'; }

// debug
void debug_out(string s) { cerr << endl; }
template<typename H, typename... T>
void debug_out(string s, H h, T... t) {
    do { cerr << s[0]; s = s.substr(1);
    } while (s.size() and s[0] != ',');
    cerr << "=" << h;
    debug_out(s, t...);
}
#ifdef DEBUG
#define debug(...) cerr << "Line(" << __LINE__ << "): "; debug_out(#__VA_ARGS__, __VA_ARGS__)
#else
#define debug(...)
#endif\end{minted}

\section{Matematica}
\subsection{crivo}
\begin{minted}{cpp}

// Crivo de Eratosthenes

// crivo - O(n log(log(n)))
// fact - O(log(n))

ll _tam_crivo;
bitset<10000010> bs;
vector<int> primos;

void crivo(ll limite){
    _tam_crivo = limite +1;
    bs.reset();bs.flip();
    bs.set(0,false); bs.set(1,false);
    for(ll i=2;i<=_tam_crivo; i++){
        if(bs.test((size_t)i)){
            for(ll j = i*i; j<=_tam_crivo;j+=i){
                bs.set((size_t)j,false);
            }
            primos.push_back((int)i);
        }
    }
}

bool eh_primo(ll N){
    if(N<_tam_crivo) return bs.test(N);
    for(int i=0;i<primos.size();i++){
        if(N%primos[i]==0){
            return false;
        }
    }
    return true;
}

vector<int> primeFactors(int N){
    vector<int> factors;
    int PF_idx = 0, PF = primos[PF_idx];
    while(N!=1 && PF*PF <= N){
        while(N%PF == 0){
            N/=PF;
            factors.push_back(PF);
        }
        PF=primos[++PF_idx];
    }
    if(N!=1) factors.push_back(N);
    return factors;
}

\end{minted}

\end{document}